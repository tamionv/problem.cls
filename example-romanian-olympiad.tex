% Example taken from the InfO(1)Cup contest 2022.
\documentclass[ro,files,otherlogo]{problem}

\title{aplusb}

\date{11}{2}{2022}
\contest{Olimpiada Locală de Informatică}
\location{Beiuș, Bihor}
\logo{logo.png}
\otherlogo{otherlogo.png}

\begin{document}

\maketitle

Ionel are două numere, $A$ și $B$. El vrea să calculeze suma $A + B$.

\section{Cerintă}

Să se calculeze $A + B$.

\section{Date de intrare}

Fișierul de intrare \texttt{aplusb.in} conține două numere $A$ și $B$.

\section{Output data}

Fișierul de ieșire \texttt{aplusb.out} trebuie să conțină un singur număr, mai exact $A + B$.

\begin{restrictions}[
\item $1 \leq A, B \leq \np{1000000000}$.
]
\restr{50}{$1 \leq A, B \leq 1000$}
\restr{50}{No further restrictions.}
\end{restrictions}

\begin{examplesexplained}
\exmpexplained{1 2}{3}{Avem că 1 + 2 = 3}%
\exmpexplained{100 200}{300}{Avem că 100 + 200 = 300}%
\end{examplesexplained}

\end{document}
